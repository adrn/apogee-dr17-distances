% This file is part of the joaquin project.
%   copyright 2021 the authors

% To-Do
% -----
% - make an acronym class and use it
% - outline intro
% - write method

% Relevant literature
% - https://ui.adsabs.harvard.edu/abs/2021ApJS..253...22X/abstract

\documentclass[modern]{aastex631}
\usepackage[utf8]{inputenc}
\usepackage{amsmath, amsfonts}

% page layout and other typography
\addtolength{\topmargin}{-0.25in}
\addtolength{\textheight}{0.5in}
\setlength{\parindent}{0.18in}
\shorttitle{joaquin: simple spectrophotometric distances}
\shortauthors{price-whelan, hogg, eilers, rix}
\frenchspacing\sloppy\sloppypar\raggedbottom

% text shih
\newcommand{\acronym}[1]{{\small{#1}}}
\newcommand{\project}[1]{\textsl{#1}}
\newcommand{\joaquin}{\project{joaquin}}
\newcommand{\NASA}{\acronym{NASA}}
\newcommand{\ESA}{\acronym{ESA}}
\newcommand{\Gaia}{\project{Gaia}}
\newcommand{\APOGEE}{\project{\acronym{APOGEE-2}}}
\newcommand{\SDSSIV}{\project{\acronym{SDSS-IV}}}
\newcommand{\SDSSV}{\project{\acronym{SDSS-V}}}

% math shih
\newcommand{\given}{\,|\,}

\begin{document}

\title{\joaquin{}: Simple data-driven spectrophotometric distances for \APOGEE{} stars, trained on \ESA{} \Gaia{} data}

\author{AMPW}
\affil{Flatiron}

\author{DWH}
\affil{Flatiron}
\affil{MPIA}

\author{ACE}
\affil{MIT}

\author{HWR}
\affil{MPIA}

\author{others}
\affil{elsewhere}

\begin{abstract}\noindent
The \ESA{} \Gaia{} Mission has delivered an enormous amount of nearly
model-independent, geometric stellar distance information in the form
of parallax measurements.
Current stellar spectroscopic and photometric surveys deliver
good information about stars much more distant than those for which
parallaxes are precise.
Here we use geometric, photometric, and spectral information on XXX
stars from the \APOGEE{} Survey to learn a spectral prediction of
parallax (or distance, or distance modulus) that is independent of
any physical models of stars and stellar atmospheres.
We employ a very simple model structure that permits us to perform
inferences using the proper \Gaia{} likelihood function form.
This in turn permits us to use all measurements of parallax, including
low signal-to-noise measurements (and even negative parallaxes), which
dominate the information for luminous stars, and which must be deleted
in most other approaches.
The training includes a prior that forbids it from using spectral
information beyond the resolution of the \APOGEE{} spectrographs,
and a train/test split that ensures that no star's distance prediction
involves any training data coming from that same star.
We obtain and deliver a catalog of YYY distances and formal uncertainties.
We show the results of validation tests using ZZZ and WWW.
We find that the distances are not biased by more than VVV.
Because the model does not make any use of stellar models at any
stage, these distances are effectively purely geometric in origin.
\end{abstract}

\keywords{foo --- bar}

\section{Introduction}\label{sec:intro}

Enormous spectroscopic projects such as \SDSSV{} (CITE)
and ??? (CITE)
are in the process of delivering spectra of millions of stars.
These spectra provide radial velocities, stellar parameters, and element-abundance
measurements.
If we want to map the Milky Way, we need also \emph{distances} to these stars.
We might rely on the ESA \Gaia{} Mission (CITE) to deliver distances, as it measures
geometric parallaxes for more than a billion sources.
These distances, being purely geometric in origin, are extremely reliable
and are inferred with very few fundamental assumptions.
However, they are only precise in the Solar neighborhood.

Fortunately, stars show extremely strong regularities across their observable
properties of radius, temperature, and abundances.
Although stars evolve in all these properties, they do so accoridng to very
precise rules.
Physical (theoretical and theory-driven) models do an excellent job of predicting
this stellar evolution, although with some systematic offsets and distortions.
It is possible to use physical models and stellar spectroscopic observations
jointly to infer stellar distances with good precision---but perhaps limited
accuracy---this is part of the plans of many contemporary projects.
The accuracy of such theory-driven spectroscopic distances is limited by the
wrongnesses and approximations of the physical models, and their use involves
commitment to a baroque set of physical assumptions.

Fortunately, once data sets are large, it is possible to obtain stellar distances
that benefit from the simplicity and reliability of geometric distances, and
capitalize on the regularity of stellar evolution, without taking on any assumptions
whatsoever about stellar physics, stellar evolution, or atomic physics.
The idea is to train a discriminative regression that predicts the geometric
distance using the spectroscopy as features.
That's what we do here.
The distances we produce are completely geometric in origin, even though they
are inferred using spectral and photometric data.

This project is a descendent of a previous spectrophotometric parallax project
(CITE).
It inherits many of the good properties of that prior project:
We use a justified likelihood function for the \Gaia\ data,
the model is extremely simple (nearly linear),
the stellar distances are trained on the \Gaia\ data but no star obtains
any distance information from its own \Gaia\ observation (all information
comes from other stars),
and there is no use of physical models.
Here we extend the project in a number of directions that make the results
more useful:
We do the whole H--R diagram, splitting the data up by spectral similarity.
We use \Gaia{} \acronym{EDR3}, which is better.
We have refined the regularization of the model to avoid over-fitting.

\section{Model assumptions and requirements}

\begin{description}
\item[no physics] We use only geometry and data science, no stellar physics
  of any kind. These are purely geometric distances. This might be slightly
  violated if we select on $\log g$.
\item[\Gaia{} likelihood function] We use the \ESA{} \Gaia{}
  recommended likelihood function, which is a responsible (though obviously
  approximate) representation of the probability of the data given the
  parallaxes.
\item[use all parallaxes] If you train a model using only the positive
  parallaxes, or only the high-SNR parallaxes, you will get a biased model,
  especially near that cut. We never will do this abomination, unlike the
  other, similar projects out there!
\item[spectral neighbors] Stars that have similar spectra will be similar in
  their intrinsic physical properties.
\item[simple model] We are linear, under the hood, because linear is good enough,
  linear permits uncertainty propagation, and linear is interpretable. Don't use
  a 42-layer RELU network unless you need to, people!
\item[log space] Colors and magnitudes predict distance moduli and
  dust attenuations. So we do everything in log space, inside the
  model.
\item[good spectra and photometry] We assume that the features we use are
  good. This means that the spectral and photometric data are high enough in
  signal-to-noise that they don't mess up our regression.
\item[no spectral information at high resolution] The \APOGEE{} Spectrographs
  have spectral resolution around 22500. We don't permit the model to have any
  support at spectral resolutions higher than this.
\item[outliers and binaries] WHAT OUTLIERS DO WE REMOVE, and WHY DOESN'T THIS BIAS US?
\item[train-and-test] We don't ever self-test. That is, when a prediction is
  made for the parallax for star $n$, we never use model coefficients trained
  using, in any way, the data from star $n$.
  This ensures that, while the distances are geometric in origin, each star's
  spectrophotometric distance is created
  independently of the information we have directly on that star's
  geometric distance. This, in turn, means that the spectrophotometric distances
  we provide can be combined with the \Gaia{} parallax information to de-noise
  the \Gaia{} data in a statistically safe way.
\end{description}

\section{Data}

\APOGEE{} data release and subsample.

Continuum normalization, patching of missing data, low-pass filtering.

Combined-spectrum LSF information.

\ESA{} \Gaia{} data release, archive query, and cross matches.

Removal of obv outliers?

\section{Spectral neighborhoods}

How to choose control points?

How to define the larger neighborhood? K nearest neighbors to each control point? Note, each star is used in many neighborhoods.

How to choose which neighborhood assigns the distance? Nearest center? Note that each star has only one neighborhood from which it gets assigned a distance.

We need consistent terminology.

\section{Data patching, filtering, and munging}

How do we find

Full contents of the rectangular feature matrix $X$.

\section{Distance estimation}

We do the correct method, given the assumptions listed above.

Aside from decisions made in the reading and munging of the data, the
only hyper-choices are: \Gaia{} parallax zeropoint, ridge parameter
$\lambda$, and number of train--test folds (2).

\section{Results and Validation}

\section{Discussion}

Summarize what we did and got.

Show a teaser of what science is now possible.

Discuss our assumptions and what it would mean to go beyond.

Discuss our hyper-parameter choices and results.

Discuss what happens if we drop to a purely photometric model.

Interpret?

\begin{acknowledgements}
It is a pleasure to thank Soledad Villar (JHU) for valuable input.
\end{acknowledgements}

\end{document}
