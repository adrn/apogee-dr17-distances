% This file is part of the joaquin project.
%   copyright 2021 the authors

% To-Do
% -----
% - make an acronym class and use it
% - outline intro
% - write method

\documentclass[modern]{aastex631}
\usepackage[utf8]{inputenc}
\usepackage{amsmath, amsfonts}

% typography
\setlength{\topmargin}{-0.50in}
\setlength{\headsep}{4ex}
\setlength{\textheight}{9.30in}
\setlength{\oddsidemargin}{0.5in}
\setlength{\textwidth}{5.50in}
\setlength{\parindent}{0.18in}
\newlength{\figurewidth}
\setlength{\figurewidth}{0.94\textwidth}
\setlength{\textfloatsep}{1ex}
\setlength{\floatsep}{1ex}
\frenchspacing\sloppy\sloppypar\raggedbottom

% text shih
\newcommand{\acronym}[1]{{\small{#1}}}
\newcommand{\project}[1]{\textsl{#1}}
\newcommand{\joaquin}{\project{joaquin}}
\newcommand{\NASA}{\acronym{NASA}}
\newcommand{\ESA}{\acronym{ESA}}
\newcommand{\Gaia}{\project{Gaia}}
\newcommand{\APOGEE}{\project{\acronym{APOGEE-2}}}

% math shih
\newcommand{\given}{\,|\,}

\begin{document}

\title{\joaquin{}: Simple data-driven spectrophotometric distances for \APOGEE{} stars, trained on \ESA{} \Gaia{} data}

\author{AMPW}
\affil{Flatiron}

\author{DWH}
\affil{Flatiron}
\affil{MPIA}

\author{ACE}
\affil{MIT}

\author{HWR}
\affil{MPIA}

\author{others}
\affil{elsewhere}

\begin{abstract}
The \ESA{} \Gaia{} Mission has delivered an enormous amount of nearly
model-independent, geometric stellar distance information in the form
of parallax measurements.
Current stellar spectroscopic and photometric surveys deliver
good information about stars much more distant than those for which
parallaxes are precise.
Here we use geometric, photometric, and spectral information on XXX
stars from the \APOGEE{} Survey to learn a spectral prediction of
parallax (or distance, or distance modulus) that is independent of
any physical models of stars and stellar atmospheres.
We employ a very simple model structure that permits us to perform
inferences using the proper \Gaia{} likelihood function form.
This in turn permits us to use all measurements of parallax, including
low signal-to-noise measurements (and even negative parallaxes), which
dominate the information for luminous stars, and which must be deleted
in most other approaches.
The training includes a prior that forbids it from using spectral
information beyond the resolution of the \APOGEE{} spectrographs,
and a train/test split that ensures that no star's distance prediction
involves any training data coming from that same star.
We obtain and deliver a catalog of YYY distances and formal uncertainties.
We show the results of validation tests using ZZZ and WWW.
We find that the distances are not biased by more than VVV.
Because the model does not make any use of stellar models at any
stage, these distances are effectively purely geometric in origin.
\end{abstract}

\keywords{foo --- bar}

\section{Introduction}\label{sec:intro}

Hello World

\section{Model assumptions and requirements}

\section{Data and pre-processing}

\section{Method}

\section{Results and Validation}

\section{Discussion}

\begin{acknowledgements}
It is a pleasure to thank Soledad Villar (JHU) for valuable input.
\end{acknowledgements}

\end{document}
